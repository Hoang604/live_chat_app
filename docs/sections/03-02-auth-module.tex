\subsection{AuthModule}
\label{subsec:auth-module}

\subsubsection{Mục đích}
\texttt{AuthModule} là trung tâm an ninh của toàn bộ hệ thống. Nó chịu trách nhiệm xác thực danh tính người dùng, tạo và quản lý các token truy cập, đồng thời cung cấp các cơ chế bảo mật nâng cao như Xác thực hai yếu tố (2FA) và đăng nhập qua mạng xã hội (Google). Module này phối hợp chặt chẽ với \texttt{UserModule} và \texttt{MailModule} để xác minh thông tin đăng nhập và quản lý các thông tin bí mật liên quan đến xác thực.

\subsubsection{Luồng Nghiệp vụ và Thành phần}

\paragraph{1. Đăng ký và Xác thực Email}
Luồng này xử lý việc tạo tài khoản mới và đảm bảo người dùng sở hữu địa chỉ email đã đăng ký.
\begin{description}
    \item[\texttt{POST /auth/register}] Endpoint công khai cho phép người dùng mới đăng ký. Nó không tự động đăng nhập mà thay vào đó:
    \begin{itemize}
        \item Kiểm tra email tồn tại.
        \item Băm mật khẩu và tạo \texttt{User} mới với trạng thái \texttt{isEmailVerified = false}.
        \item Tạo một token xác thực duy nhất, lưu vào Redis với thời gian sống 15 phút.
        \item Gửi email chào mừng chứa link xác thực (với token) cho người dùng qua \texttt{MailModule}.
    \end{itemize}
    \item[\texttt{GET /auth/verify-email}] Endpoint được truy cập từ link trong email. Nó đọc token từ query params, tìm \texttt{userId} tương ứng trong Redis, sau đó cập nhật trạng thái \texttt{isEmailVerified = true} cho user và xóa token khỏi Redis.
    \item[\texttt{POST /auth/resend-verification}] Cho phép người dùng yêu cầu gửi lại email xác thực nếu họ không nhận được hoặc token hết hạn.
\end{description}

\paragraph{2. Đăng nhập và Quản lý Session}
Luồng này xử lý việc đăng nhập bằng email/mật khẩu và vòng đời của token.
\begin{description}
    \item[\texttt{POST /auth/login}] Endpoint đăng nhập chính, được bảo vệ bởi \texttt{LocalAuthGuard}. Đây là nơi chứa logic rẽ nhánh quan trọng:
    \begin{itemize}
        \item Nếu người dùng \textbf{không} bật 2FA, nó sẽ cấp phát bộ token đầy đủ (Access Token, Refresh Token) và cài đặt \texttt{refresh\_token} vào HttpOnly cookie.
        \item Nếu người dùng \textbf{có} bật 2FA, nó sẽ không cấp token đầy đủ. Thay vào đó, nó tạo một \texttt{partial token} tạm thời, cài đặt vào cookie \texttt{2fa\_partial\_token}, và trả về lỗi \texttt{401 Unauthorized} với mã lỗi đặc biệt là \texttt{2FA\_REQUIRED}. Frontend sẽ dựa vào mã lỗi này để hiển thị màn hình nhập mã OTP.
    \end{itemize}
    \item[\texttt{GET /auth/refresh}] Endpoint để làm mới access token, được bảo vệ bởi \texttt{RefreshTokenGuard}. Nó thực hiện cơ chế \textit{Refresh Token Rotation}: xác thực refresh token cũ, nếu hợp lệ, tạo ra một cặp token mới và thay thế token cũ trong CSDL.
    \item[\texttt{POST /auth/logout}] Đăng xuất khỏi phiên hiện tại bằng cách xóa refresh token tương ứng khỏi CSDL và xóa cookie trên trình duyệt.
    \item[\texttt{POST /auth/logout-all}] Đăng xuất khỏi tất cả các thiết bị. Nó xóa tất cả refresh token của người dùng và cập nhật trường \texttt{tokensValidFrom} trên entity \texttt{User} để vô hiệu hóa tất cả access token đã cấp trước đó.
    \item[\texttt{POST /auth/change-password}] Cho phép người dùng đã xác thực thay đổi mật khẩu. Sau khi đổi thành công, nó sẽ tự động gọi \texttt{logoutAll} để tăng cường bảo mật.
\end{description}

\paragraph{3. Đăng nhập qua Google (OAuth2)}
Luồng này cho phép người dùng đăng nhập hoặc đăng ký nhanh chóng bằng tài khoản Google.
\begin{description}
    \item[\texttt{GET /auth/google}] Endpoint khởi đầu, chuyển hướng người dùng đến trang đăng nhập của Google để xác thực.
    \item[\texttt{GET /auth/google/callback}] Endpoint mà Google sẽ gọi lại sau khi người dùng xác thực thành công. Logic bên trong sẽ:
    \begin{itemize}
        \item Nhận thông tin profile từ Google.
        \item Gọi \texttt{authService.validateOAuthUser} để tìm hoặc tạo user tương ứng.
        \item Nếu user có bật 2FA, tạo \texttt{partial token} và chuyển hướng đến trang nhập mã 2FA của frontend.
        \item Nếu không, tạo một \textbf{mã sử dụng một lần (one-time code)}, lưu vào Redis, và chuyển hướng người dùng về trang callback của frontend kèm theo mã này.
    \end{itemize}
    \item[\texttt{POST /auth/exchange-code}] Frontend sau khi nhận được one-time code sẽ gọi đến endpoint này. Backend sẽ xác thực mã trong Redis, nếu hợp lệ thì xóa mã và cấp phát bộ token đầy đủ cho người dùng, hoàn tất quá trình đăng nhập.
\end{description}

\paragraph{4. Xác thực hai yếu tố (2FA)}
Luồng 2FA cung cấp một lớp bảo mật bổ sung, yêu cầu mã OTP từ ứng dụng xác thực.
\begin{description}
    \item[\texttt{POST /2fa/generate}] Endpoint cho người dùng đã đăng nhập để bắt đầu quá trình cài đặt 2FA. Nó tạo ra một secret key và mã QR. Secret này được mã hóa và lưu tạm thời vào một HttpOnly cookie (\texttt{2fa\_secret}) có hiệu lực 5 phút, thay vì lưu ngay vào CSDL.
    \item[\texttt{POST /2fa/turn-on}] Hoàn tất việc bật 2FA. Nó đọc secret từ cookie, xác thực mã OTP người dùng nhập. Nếu đúng, nó sẽ lưu vĩnh viễn secret vào CSDL và tạo các mã khôi phục (recovery codes).
    \item[\texttt{POST /2fa/authenticate}] Endpoint được sử dụng sau khi đăng nhập bằng mật khẩu (hoặc Google) nếu tài khoản có bật 2FA. Nó được bảo vệ bởi một Guard đặc biệt (\texttt{AuthGuard('2fa-partial')}) chỉ cho phép các request chứa \texttt{partial token}. Nếu mã OTP hợp lệ, nó sẽ cấp phát bộ token đầy đủ.
    \item[\texttt{POST /2fa/turn-off}] Vô hiệu hóa 2FA, yêu cầu một mã OTP cuối cùng để xác nhận hành động.
\end{description}