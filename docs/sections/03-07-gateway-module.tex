\subsection{GatewayModule}
\label{subsec:gateway-module}

\subsubsection{Mục đích và Kiến trúc}

\texttt{GatewayModule} là trung tâm điều phối cho toàn bộ giao tiếp hai chiều, thời gian thực (real-time) của hệ thống. Tuy nhiên, nó không trực tiếp xử lý logic nghiệp vụ. Thay vào đó, nó được thiết kế theo một kiến trúc bất đồng bộ, hiệu năng cao, đóng vai trò là một lớp giao tiếp (communication layer) tinh gọn.

Trách nhiệm chính của module này là:
\begin{itemize}
    \item Quản lý vòng đời của các kết nối WebSocket (kết nối, ngắt kết nối).
    \item Tiếp nhận các sự kiện từ client và \textbf{ngay lập tức chuyển đổi chúng thành các sự kiện nội bộ} bằng cách sử dụng \texttt{EventEmitter2}. Điều này giúp gateway không bao giờ bị block bởi các tác vụ tốn thời gian như truy vấn CSDL.
    \item Cung cấp một API để các service khác trong backend có thể gửi các sự kiện real-time đến các client cụ thể.
    \item Lắng nghe các kênh Redis Pub/Sub để nhận thông điệp từ các server instance khác, cho phép hệ thống mở rộng theo chiều ngang.
\end{itemize}

\subsubsection{Bảo mật: WsJwtAuthGuard}
Toàn bộ gateway được bảo vệ bởi một Guard tùy chỉnh là \texttt{WsJwtAuthGuard}. Guard này có một logic đặc biệt, cho phép hai loại kết nối:
\begin{itemize}
    \item \textbf{Kết nối đã xác thực (Authenticated):} Nếu client (Agent Dashboard) gửi lên một JWT hợp lệ trong phần \texttt{handshake.auth.token}, Guard sẽ xác thực token, lấy thông tin người dùng và gắn vào \texttt{client.data.user}. Các trình xử lý sự kiện sau đó có thể sử dụng thông tin này.
    \item \textbf{Kết nối chưa xác thực (Unauthenticated):} Nếu client (Widget của visitor) không gửi token, Guard sẽ \textbf{vẫn cho phép kết nối}. Điều này cho phép các visitor ẩn danh có thể kết nối và gửi tin nhắn.
\end{itemize}

\subsubsection{Phân tích các Luồng và Phương thức Quan trọng}

\paragraph{Vòng đời và Giao tiếp liên-server}
\begin{itemize}
    \item \texttt{handleConnection(client)}: Ghi log khi có client mới kết nối.
    \item \texttt{handleDisconnect(client)}: Khi một client ngắt kết nối, nó sẽ gọi \texttt{realtimeSessionService.deleteVisitorSession} để dọn dẹp session của visitor đó khỏi Redis.
    \item \texttt{afterInit()}: Một lifecycle hook cực kỳ quan trọng. Khi gateway khởi động, nó sẽ đăng ký lắng nghe một kênh Redis Pub/Sub (\texttt{new\_message\_channel}). Khi một server khác trong cụm xử lý một sự kiện và cần thông báo cho gateway này, nó sẽ publish một message vào kênh Redis đó. Phương thức này đảm bảo gateway có thể nhận và xử lý các thông điệp liên-server.
\end{itemize}

\paragraph{Xử lý Sự kiện từ Client (Client -> Server)}
\begin{itemize}
    \item \texttt{@SubscribeMessage('identify')} và \texttt{@SubscribeMessage('sendMessage')}: Đây là hai trình xử lý sự kiện quan trọng nhất từ widget. Chúng \textbf{không} thực hiện bất kỳ logic nghiệp vụ nào. Thay vào đó, chúng chỉ đóng gói payload nhận được từ client và phát ra một sự kiện nội bộ tương ứng (\texttt{visitor.identified} hoặc \texttt{visitor.message.received}) bằng \texttt{EventEmitter2}. Các module khác (như \texttt{EventConsumerModule}) sẽ lắng nghe và xử lý các sự kiện này một cách bất đồng bộ.
    
    \item \texttt{@SubscribeMessage('joinProjectRoom')} và \texttt{@SubscribeMessage('leaveProjectRoom')}: Các trình xử lý này cho phép Agent Dashboard tham gia hoặc rời khỏi một "room" Socket.IO tương ứng với một dự án (ví dụ: \texttt{project:123}). Điều này cho phép việc gửi sự kiện đến tất cả các agent đang xem một dự án trở nên dễ dàng.
    
    \item \texttt{@SubscribeMessage('visitorIsTyping')} và \texttt{@SubscribeMessage('updateContext')}: Các trình xử lý này nhận trạng thái từ widget và phát (broadcast) các sự kiện tương ứng đến room của dự án, cho phép các agent dashboard nhận được cập nhật theo thời gian thực.
\end{itemize}

\paragraph{API Gửi Sự kiện từ Server (Server -> Client)}
Đây là các phương thức public mà các service khác trong backend sử dụng để giao tiếp với client.
\begin{itemize}
    \item \texttt{prepareSocketForVisitor(...)}: Một phương thức quan trọng được gọi bởi \texttt{EventConsumerModule} sau khi một visitor được định danh thành công. Nó thực hiện các việc: gắn \texttt{projectId}, \texttt{visitorUid} vào đối tượng socket; và quan trọng nhất là gửi sự kiện \texttt{conversationHistory} chứa các tin nhắn cũ xuống cho widget.
    
    \item \texttt{sendReplyToVisitor(socketId, message)}: Gửi tin nhắn trả lời của agent đến một visitor cụ thể thông qua \texttt{socketId} đã được tra cứu từ Redis.
    
    \item \texttt{sendAgentTypingToVisitor(...)}: Gửi sự kiện "agent đang gõ" đến một visitor cụ thể.
    
    \item \texttt{visitorMessageSent(socketId, data)}: Gửi một sự kiện xác nhận ngược lại cho visitor rằng tin nhắn của họ đã được server nhận và xử lý thành công.
\end{itemize}