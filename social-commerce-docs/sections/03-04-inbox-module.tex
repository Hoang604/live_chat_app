\subsection{InboxModule}
\label{subsec:inbox-module}

\subsubsection{Mục đích}

\texttt{InboxModule} là trung tâm nghiệp vụ của toàn bộ ứng dụng. Nó chịu trách nhiệm quản lý toàn bộ luồng giao tiếp giữa khách truy cập (visitors) và nhân viên hỗ trợ (agents). Module này chứa đựng các thực thể, logic nghiệp vụ và các API endpoint cần thiết để tạo ra một trải nghiệm hộp thư hợp nhất (unified inbox), nơi agents có thể xem và trả lời tất cả các cuộc hội thoại từ các dự án mà họ là thành viên.

\subsubsection{Phân tích Entities}
\label{subsubsec:inbox-entities}

Các entities trong module này định hình nên cấu trúc dữ liệu cốt lõi của một hệ thống chat.

\paragraph{\texttt{Visitor.entity.ts}}
\begin{itemize}
    \item \textbf{Mục đích:} Đại diện cho một khách truy cập duy nhất trên website của một dự án.
    \item \textbf{Các cột quan trọng:}
    \begin{itemize}
        \item \texttt{visitorUid}: Định danh công khai (UUID) được lưu trữ trên trình duyệt của khách truy cập.
        \item \texttt{projectId}: Khóa ngoại, liên kết visitor với một \texttt{Project} cụ thể.
        \item \texttt{displayName}: Tên định danh thân thiện do hệ thống tự tạo (ví dụ: "Visitor \#abc123").
        \item \texttt{metadata}: Trường \texttt{jsonb} linh hoạt để lưu trữ thông tin ngữ cảnh bổ sung.
        \item \texttt{lastSeenAt}: Dấu thời gian (timestamp) ghi nhận lần cuối cùng visitor hoạt động.
        \item \texttt{conversations}: Mối quan hệ một-nhiều (\texttt{OneToMany}) với entity \texttt{Conversation}.
    \end{itemize}
\end{itemize}

\paragraph{\texttt{Conversation.entity.ts}}
\begin{itemize}
    \item \textbf{Mục đích:} Đại diện cho một luồng hội thoại (thread) hoàn chỉnh giữa một \texttt{Visitor} và các nhân viên hỗ trợ.
    \item \textbf{Các cột quan trọng:}
    \begin{itemize}
        \item \texttt{project} và \texttt{visitor}: Hai mối quan hệ nhiều-một (\texttt{ManyToOne}) xác định cuộc hội thoại này thuộc về dự án nào và của visitor nào.
        \item \texttt{status}: Trường \texttt{enum} (\texttt{OPEN}, \texttt{CLOSED}) để quản lý trạng thái của cuộc hội thoại.
        \item \texttt{lastMessageSnippet} \& \texttt{lastMessageTimestamp}: Các trường dữ liệu được phi chuẩn hóa (denormalized) để tối ưu hiệu năng khi hiển thị danh sách hội thoại mà không cần JOIN sang bảng \texttt{messages}.
        \item \texttt{unreadCount}: Bộ đếm số lượng tin nhắn chưa đọc từ phía visitor.
        \item \texttt{messages}: Mối quan hệ một-nhiều (\texttt{OneToMany}) với entity \texttt{Message}.
    \end{itemize}
\end{itemize}

\paragraph{\texttt{Message.entity.ts}}
\begin{itemize}
    \item \textbf{Mục đích:} Đại diện cho một tin nhắn ("bong bóng chat") duy nhất trong một cuộc hội thoại.
    \item \textbf{Các cột quan trọng:}
    \begin{itemize}
        \item \texttt{conversation}: Mối quan hệ nhiều-một (\texttt{ManyToOne}) để liên kết tin nhắn với cuộc hội thoại mẹ.
        \item \texttt{content}: Chứa nội dung văn bản của tin nhắn.
        \item \texttt{attachments}: Trường \texttt{jsonb} để lưu thông tin về các tệp đính kèm (hình ảnh, file).
        \item \texttt{fromCustomer}: Cờ \texttt{boolean} cho biết tin nhắn đến từ khách truy cập hay từ nhân viên hỗ trợ, giúp UI quyết định vị trí hiển thị.
        \item \texttt{status}: Trường \texttt{enum} (\texttt{SENDING}, \texttt{SENT}, \texttt{DELIVERED}, \texttt{FAILED}) để theo dõi trạng thái gửi của tin nhắn.
    \end{itemize}
\end{itemize}

\subsubsection{Phân tích Services}
\label{subsubsec:inbox-services}

Các service trong module này được phân tách rõ ràng theo từng thực thể, mỗi service chịu trách nhiệm cho một domain nghiệp vụ cụ thể.

\paragraph{\texttt{inbox/services/visitor.service.ts}}
\begin{itemize}
    \item \textbf{Mục đích:} Cổng giao tiếp cho tất cả các thao tác liên quan đến thực thể \texttt{Visitor}.
    \item \textbf{Các phương thức quan trọng:}
    \begin{itemize}
        \item \texttt{findOrCreateByUid(...)}: Phương thức giao tác (transactional) được \texttt{EventConsumerService} sử dụng để tìm hoặc tạo mới một visitor. Nó cũng cập nhật \texttt{lastSeenAt} cho visitor cũ.
        \item \texttt{findByUid(visitorUid)}: Phương thức truy vấn nhanh, không giao tác, cho phép \texttt{EventsGateway} kiểm tra sự tồn tại của visitor.
        \item \texttt{getVisitorById(visitorId)}: Lấy thông tin chi tiết của một visitor, được sử dụng bởi \texttt{InboxController} để hiển thị trên dashboard.
    \end{itemize}
\end{itemize}

\paragraph{\texttt{inbox/services/conversation.service.ts}}
\begin{itemize}
    \item \textbf{Mục đích:} Quản lý vòng đời và dữ liệu của các luồng hội thoại.
    \item \textbf{Các phương thức quan trọng:}
    \begin{itemize}
        \item \texttt{findOrCreateByVisitorId(...)}: Phương thức giao tác để tìm hoặc tạo một cuộc hội thoại mới cho một visitor.
        \item \texttt{updateLastMessage(...)}: Thực hiện phi chuẩn hóa dữ liệu bằng cách cập nhật \texttt{lastMessageSnippet}, \texttt{lastMessageTimestamp}, và tăng \texttt{unreadCount} của một cuộc hội thoại khi có tin nhắn mới.
        \item \texttt{listByProject(...)}: Cung cấp chức năng chính cho Agent Dashboard. Nó sử dụng \texttt{QueryBuilder} để xây dựng một truy vấn phức tạp, đáng chú ý là có một \texttt{INNER JOIN} đến bảng \texttt{project\_members} để đảm bảo một agent chỉ có thể thấy các cuộc hội thoại từ các dự án mà họ là thành viên.
        \item \texttt{updateStatus(...)} và \texttt{markAsRead(...)}: Các phương thức cho phép agent thay đổi trạng thái của cuộc hội thoại (open/closed) hoặc đánh dấu là đã đọc (reset \texttt{unreadCount}). Cả hai đều có bước kiểm tra quyền thành viên dự án.
        \item \texttt{getHistoryByVisitorId(...)}: Tìm một cuộc hội thoại đang mở của visitor và tải kèm toàn bộ lịch sử tin nhắn để gửi lại cho widget khi visitor kết nối lại.
        \item \texttt{handleAgentTyping(...)}: Hiện thực hóa tính năng "agent đang gõ" bằng cách tìm socket của visitor và phát sự kiện real-time.
    \end{itemize}
\end{itemize}

\paragraph{\texttt{inbox/services/message.service.ts}}
\begin{itemize}
    \item \textbf{Mục đích:} Chịu trách nhiệm cho tất cả các thao tác liên quan đến các tin nhắn riêng lẻ.
    \item \textbf{Các phương thức quan trọng:}
    \begin{itemize}
        \item \texttt{createMessageAndVerifySent(...)}: Phương thức giao tác, được \texttt{EventConsumerService} gọi để lưu một tin nhắn mới từ visitor vào CSDL.
        \item \texttt{sendAgentReply(...)}: Phương thức cốt lõi cho luồng trả lời của agent. Nó thực hiện một chuỗi hành động: kiểm tra quyền, tạo và lưu tin nhắn, tìm socket của visitor, và gửi tin nhắn real-time qua gateway.
        \item \texttt{listByConversation(...)}: Lấy danh sách tin nhắn có phân trang (sử dụng cursor-based pagination) cho một cuộc hội thoại cụ thể, có kiểm tra quyền thành viên.
    \end{itemize}
\end{itemize}


\subsubsection{Phân tích Controller}
\label{subsubsec:inbox-controller}

\paragraph{\texttt{inbox.controller.ts}}
\begin{itemize}
    \item \textbf{Mục đích:} Là cổng giao tiếp API cho Agent Dashboard. Toàn bộ controller được bảo vệ bởi \texttt{JwtAuthGuard} và \texttt{RolesGuard}, yêu cầu người dùng phải là thành viên của dự án (\texttt{AGENT} hoặc \texttt{MANAGER}).
    \item \textbf{Các phương thức quan trọng:}
    \begin{itemize}
        \item \texttt{GET /inbox/conversations}: Lấy danh sách các cuộc hội thoại cho một dự án, có phân trang và lọc theo trạng thái.
        \item \texttt{POST /inbox/conversations/:id/messages}: Để một agent gửi tin nhắn trả lời.
        \item \texttt{PATCH /inbox/conversations/:id}: Cập nhật trạng thái của một cuộc hội thoại (open/closed) hoặc đánh dấu là đã đọc.
        \item \texttt{GET /inbox/conversations/:id/messages}: Lấy lịch sử tin nhắn của một cuộc hội thoại, có phân trang.
        \item \texttt{POST /inbox/conversations/:id/typing}: Phục vụ cho tính năng "agent đang gõ".
        \item \texttt{GET /inbox/visitors/:id}: Lấy thông tin chi tiết về một visitor, bao gồm cả metadata.
    \end{itemize}
\end{itemize}