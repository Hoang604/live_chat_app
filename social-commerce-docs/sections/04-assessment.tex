\section{Đánh giá và Lộ trình Tương lai}
\label{sec:assessment}

Phần này cung cấp một cái nhìn tổng hợp và chuyên môn về chất lượng của hệ thống hiện tại, đồng thời vạch ra một lộ trình rõ ràng cho các cải tiến trong tương lai, dựa trên cả phân tích mã nguồn backend và các ghi chép về quá trình phát triển, sửa lỗi, và phân tích UI/UX.

\subsection{Đánh giá Kiến trúc Backend}
\label{subsec:backend-assessment}

Kiến trúc backend hiện tại thể hiện một sự trưởng thành và độ tin cậy cao, dựa trên các quyết định thiết kế chiến lược.

\begin{itemize}
    \item \textbf{Kiến trúc Bất đồng bộ Toàn diện:}
    Hệ thống đã triển khai thành công một luồng xử lý sự kiện bất đồng bộ phức tạp và hiệu quả. Thay vì một kết nối trực tiếp, luồng hoạt động thực tế là: \texttt{Gateway -> EventEmitter -> EventListener -> SQS Producer -> SQS Queue -> EventConsumer -> Redis Pub/Sub -> Gateway}. Luồng đi này cho thấy một sự tách biệt (decoupling) triệt để, giúp cho lớp giao tiếp real-time (Gateway) luôn có hiệu năng cao, đồng thời đảm bảo không có tin nhắn nào bị mất và hệ thống có khả năng mở rộng tốt.

    \item \textbf{Phân quyền theo Vai trò (RBAC) Mạnh mẽ:}
    Hệ thống đã vượt qua được điểm yếu của mô hình sở hữu đơn lẻ. Bằng cách sử dụng một bảng trung gian \texttt{ProjectMember} và một \texttt{RolesGuard} tùy chỉnh, logic phân quyền giờ đây được tập trung, rõ ràng và an toàn. Việc phân định rạch ròi vai trò \texttt{MANAGER} và \texttt{AGENT} là nền tảng vững chắc cho các tính năng cộng tác.

    \item \textbf{Tính Toàn vẹn Dữ liệu Cao:}
    Việc sử dụng \textbf{database transaction} trong các nghiệp vụ ghi dữ liệu phức tạp (như xử lý tin nhắn mới hoặc chấp nhận lời mời) là một điểm cộng rất lớn, đảm bảo dữ liệu luôn ở trạng thái nhất quán.

    \item \textbf{Cấu trúc Module hóa Rõ ràng:}
    Việc phân chia các domain nghiệp vụ thành các module NestJS riêng biệt (User, Auth, Project, Inbox, v.v.) giúp cho codebase có tổ chức, dễ hiểu và dễ bảo trì.
\end{itemize}

\subsection{Tổng kết Quá trình Phát triển Gần đây}
\label{subsec:development-summary}

Các ghi chép về quá trình sửa lỗi cho thấy một sự trưởng thành trong việc gỡ rối và ổn định hóa hệ thống. Hai ví dụ tiêu biểu:

\begin{itemize}
    \item \textbf{Fix lỗi Conversation Status:} Một lỗi rất tinh vi đã được phát hiện và sửa chữa, trong đó việc so sánh ID dạng chuỗi và dạng số (\texttt{'2' === 2}) trong mã frontend đã thất bại. Việc này cho thấy quá trình debug đã đi sâu vào cả frontend và backend để tìm ra nguyên nhân gốc rễ và áp dụng giải pháp chuẩn xác (sử dụng \texttt{Number()}).
    
    \item \textbf{Bổ sung API theo nhu cầu:} Việc tạo ra endpoint \texttt{GET /inbox/visitors/:id} là một ví dụ điển hình cho việc phát triển lặp, trong đó backend đã phản hồi nhanh chóng để đáp ứng nhu cầu từ phía frontend (hiển thị thông tin visitor), hoàn thiện một tính năng còn dang dở.
\end{itemize}

\subsection{Phân tích UI/UX và Lộ trình Tương lai}
\label{subsec:ui-ux-roadmap}

Các phân tích về giao diện người dùng chỉ ra rằng trong khi backend đã rất mạnh mẽ, frontend vẫn còn nhiều cơ hội để cải thiện trải nghiệm người dùng và bắt kịp với các tiêu chuẩn của ngành.

\subsubsection{Điểm yếu của UI hiện tại}
\begin{itemize}
    \item \textbf{Trang Inbox:} Thiếu các tính năng quan trọng như tìm kiếm, bộ lọc nâng cao, timestamp cho tin nhắn, avatar, và message grouping. Giao diện soạn tin nhắn còn quá đơn giản, chưa hỗ trợ file đính kèm hay định dạng văn bản.
    \item \textbf{Trang Settings:} Layout chưa responsive, các form còn cơ bản, thiếu các tính năng nâng cao như xem trước avatar, chỉ báo độ mạnh mật khẩu, hay xem trước widget. Các thẻ Project còn đơn điệu, chưa hiển thị các chỉ số thống kê quan trọng.
\end{itemize}

\subsubsection{Lộ trình Nâng cấp Đề xuất}
Một lộ trình phát triển cho frontend đã được vạch ra, chia thành các giai đoạn rõ ràng:
\begin{description}
    \item[Phase 1 (Nền tảng):] Nâng cấp hệ thống màu sắc, animation, và các component cơ bản như Avatar, Header.
    \item[Phase 2 (Trải nghiệm Cốt lõi):] Tập trung vào các cải tiến quan trọng nhất cho Inbox như message grouping, search/filter, và trình soạn thảo đa phương tiện.
    \item[Phase 3 (Thông tin):] Làm giàu thông tin hiển thị, bao gồm panel chi tiết về visitor, timestamp và status indicator.
    \item[Phase 4 (Tính năng Nâng cao):] Bổ sung các tính năng chuyên nghiệp như hỗ trợ file đính kèm, trả lời soạn sẵn (canned responses), và phím tắt.
    \item[Phase 5 (Cộng tác \& Hiệu năng):] Xây dựng các tính năng làm việc nhóm như phân công hội thoại, và tối ưu hiệu năng cho quy mô lớn.
\end{description}

\subsection{Đề xuất Cải tiến cho Backend}
\label{subsec:future-improvements}

Dựa trên các phân tích trên, các cải tiến tiếp theo cho backend nên tập trung vào việc hỗ trợ các tính năng frontend mới và tiếp tục tối ưu hóa.

\begin{itemize}
    \item \textbf{Hỗ trợ các Tính năng Nâng cao cho Inbox:}
    \begin{itemize}
        \item \textbf{Conversation Assignment:} Xây dựng logic và API cho phép các \texttt{MANAGER} có thể gán một cuộc hội thoại cho một agent cụ thể.
        \item \textbf{Saved Replies:} Tạo các endpoint CRUD để quản lý các mẫu câu trả lời soạn sẵn, giúp tăng năng suất cho agent.
        \item \textbf{Attachments:} Hiện thực hóa logic lưu trữ file (ví dụ: qua S3) và cập nhật API gửi tin nhắn để hỗ trợ file đính kèm.
    \end{itemize}
    
    \item \textbf{Tối ưu hóa Truy vấn CSDL:}
    \begin{itemize}
        \item \textbf{Mô tả:} Tiếp tục rà soát các service nghiệp vụ để xác định các cơ hội tối ưu hóa truy vấn, đặc biệt là ở các API trả về danh sách (listing APIs).
        \item \textbf{Hành động cụ thể:} Sử dụng TypeORM QueryBuilder để xây dựng các truy vấn phức tạp hơn, kết hợp nhiều phép JOIN để lấy tất cả dữ liệu cần thiết chỉ trong một lần gọi, giảm số lượng round-trip đến CSDL.
    \end{itemize}
\end{itemize}